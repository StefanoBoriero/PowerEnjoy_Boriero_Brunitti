\documentclass{article}
\usepackage{graphicx}
\graphicspath{{Images/}}
\usepackage{hyperref}
\usepackage{adjustbox}
\usepackage{blindtext}
\usepackage{enumitem}
\usepackage{float}
\usepackage{array}
\usepackage{amsmath}

\begin{document}
\title{\textbf{{\Huge PowerEnJoy project: Project Plan Document}}}
\author{\begin{large}
Boriero Stefano  876106  
Brunitti Simone   875039
\end{large} }


\maketitle
\includegraphics[scale=0.5]{Logo_Politecnico_Milano.png} 
\newpage
\tableofcontents

\newpage

\section{Introduction}
\subsection{Revision History}
\subsection{Purpose and Scope}
The purpose of this document is to present the Project Plan for PowerEnJoy platform. We will first estimate the size of the project in all its parts. Upon this estimation, we will estimate the cost of the project exploiting COCOMO II estimation parameters and equations.\\
After that we will schedule tasks and allocate resource to them. To present this we are using Gantt diagrams.\\
Finally we're going to analyze possible risks we migth face during development, providing preventive solutions.
\subsection{Definitions, Acronyms, Abbreviations}
\subsubsection{Definitions}
\subsubsection{Acronyms}
\begin{itemize}
	\item FP: Function Points.
	\item ILF: Internal logic file
	\item ELF: External logic file.
	\item EI: External Input.
	\item EO: External Output.
	\item EQ: External Inquiries.
	\item UI: User Interface.
	\item GPS: Global Positioning System.
\end{itemize}
\subsubsection{Abbreviations}
\subsection{Reference Documents}
\begin{itemize}
	\item \href{https://github.com/StefanoBoriero/PowerEnjoy_Boriero_Brunitti/blob/master/Assignments%20AA%202016-2017.pdf}{Assignment}
\item \href{https://github.com/StefanoBoriero/PowerEnjoy_Boriero_Brunitti/blob/master/releases/RASD_v1.md}{RASD}
\item \href{https://github.com/StefanoBoriero/PowerEnjoy_Boriero_Brunitti/blob/master/releases/DD_v1.1.pdf}{Desing Document}
	\item \href{https://github.com/StefanoBoriero/PowerEnjoy_Boriero_Brunitti/blob/master/releases/ITPD_v1.pdf} Integration Testing Project Document
	\item \href{http://sunset.usc.edu/research/COCOMOII/expert_cocomo/drivers.html}{COCOMO II Cost Drivers}
	\item \href{http://csse.usc.edu/csse/research/COCOMOII/cocomo2000.0/CII_modelman2000.0.pdf}{COCOMO II Documentation}
\end{itemize} 
\newpage
\section{Project Size and Cost Estimation}

\subsection{Size estimation: function points}
For estimating function points we will use the following tables:\\
Internal Logic Files and External Logic Files \\\\
\begin{tabular}{|p{3cm}|p{1cm}|p{1cm}|p{1cm}|}
\hline
& \multicolumn{3}{l|}{Data Elements}\\
\hline
\textit{Record Elements} & \textit{1-19} & \textit{20-50} & \textit{51+} \\
\hline
1 & Low & Low & Avg\\
\hline
2-5 & Low & Avg & High\\
\hline
6+ & Avg & High & High\\
\hline
\end{tabular}
\\\\\\
External Output and External Inquiry
\\\\
\begin{tabular}{|p{3cm}|p{1cm}|p{1cm}|p{1cm}|}
\hline
& \multicolumn{3}{l|}{Data Elements}\\
\hline
\textit{Record Elements} & \textit{1-5} & \textit{6-19} & \textit{20+} \\
\hline
0-1 & Low & Low & Avg\\
\hline
2-3 & Low & Avg & High\\
\hline
4+ & Avg & High & High\\
\hline
\end{tabular}
\\\\\\
External Input
\\\\
\begin{tabular}{|p{3cm}|p{1cm}|p{1cm}|p{1cm}|}
\hline
& \multicolumn{3}{l|}{Data Elements}\\
\hline
\textit{Record Elements} & \textit{1-4} & \textit{5-15} & \textit{16+} \\
\hline
0-1 & Low & Low & Avg\\
\hline
2-3 & Low & Avg & High\\
\hline
4+ & Avg & High & High\\
\hline
\end{tabular}
\\\\\\
UFP Complexity Weights
\\\\
\begin{tabular}{|p{3cm}|p{1cm}|p{1cm}|p{1cm}|}
\hline
& \multicolumn{3}{l|}{Data Elements}\\
\hline
\textit{Function Type} & \textit{Low} & \textit{Average} & \textit{High} \\
\hline
Internal Logic Files & 7 & 10 & 15\\
\hline
External Logic Files & 5 & 7 & 10\\
\hline
External Inputs & 3 & 4 & 6\\
\hline
External Outputs & 4 & 5 & 7\\
\hline
External Inquiries & 3 & 4 & 6\\
\hline
\end{tabular}

\subsubsection{Internal Logic Files}
In order to provide its functionalities, PowerEnjoy will need to store data and information.\\
The main information it need to store is about vehicles. The relevant attributes needed are 
\begin{itemize}
\item Vehicle Id
\item Plate number
\item Status
\item BatteryLevel
\item Position
\end{itemize}

Another fundamental table is the one about rides, containing this information:
\begin{itemize}
\item Starting point
\item Starting battery
\item Number of passengers
\item Final position
\item Final battery
\end{itemize}

To complete the set of tables oriented to the actual sharing functionality, the system will have a Reservation table with the following information:
\begin{itemize}
\item User id
\item Vehicle id
\item Starting time
\item Unlocked car
\end{itemize}

For assistance list related functionalities the system needs an assistance table to store:
\begin{itemize}
\item Vehicle id
\item Fault
\item Status
\item Employee in charge
\end{itemize}

Employee's information are stored as well:
\begin{itemize}
\item Employee id
\item Name
\item Surname
\item District
\item Job
\end{itemize}

Then the system needs all usual information about users:
\begin{itemize}
\item User id
\item Username
\item Name
\item Surname
\item Driving license
\end{itemize}

For security issues, payment information are stored in a different table:
\begin{itemize}
\item User id
\item Payment info
\end{itemize}


Safe areas are store in a two-level structure. The first one containig:
\begin{itemize}
\item SafeArea id
\item Type
\end{itemize}


The second one containing:
\begin{itemize}
\item SafeArea id
\item Position
\end{itemize}


The following table sums up ILF's complexity and FP associated
\\
\begin{tabular}{|p{5cm}|p{3cm}|p{2cm}|}
\hline
\textbf{ILF} & \textbf{Complexity} & \textbf{FP} \\
\hline

Login data & Low & 7\\
User data & Low & 7\\
Employee data & Low & 7\\
Password & Average & 10\\
Safe areas & Low & 7\\
Vehicle & Low & 7\\
Reservation & Low & 7\\
Ride & Average & 10\\
Assistance & Low & 7\\
\hline
\multicolumn{2}{|l|}{\textbf{Total}} & 69\\
\hline
\end{tabular}
\subsubsection{External Logic Files(ELFs)}
The first external data source of PowerEnJoy is Google Maps API. The interactions between our system and this API are the following:
\begin{itemize}[noitemsep]
\item \textbf{Map Displaying}: The API must interact with the customer's terminal device in order to display a map showing the nearest cars and their distance in an area selected by the user. This first interaction is made quite simple by Google Maps API, so we will give it an average complexity
\item \textbf {Drawing Tools}: The API must interact with the employee system to allow the employee to draw new safe areas and see the existing ones. This interaction is quite complex, so we will give it an high complexity.
\end{itemize}
The following table sums up ELF complexity and FP associated\\\\
\begin{tabular}{|p{5cm}|p{3cm}|p{2cm}|}
\hline
\textbf{ELF} & \textbf{Complexity} & \textbf{FP} \\
\hline

Map Displaying & Average & 7\\
Drawing Tools & High & 10\\
\hline
\multicolumn{2}{|l|}{\textbf{Total}} & 17\\
\hline
\end{tabular}
\subsubsection{External Inputs (EIs)}
In this section we will list all the different kind of interactions that PowerEnJoy has with its users.
First, we list the interactions that any user can have with the platform: 
\begin{itemize}[noitemsep]
\item Login/Logout: simple operations, involve only the Account Controller. Points = 2x3FPs.
\end{itemize}
Then, we list the interactions between the customer and the platform:
\begin{itemize}[noitemsep]
\item Register a new account: Since this feature requires some data to be checked, we give it an Average complexity. Points = 4FPs.
\item Rent a car : This is a key feature and a highly complex one, involving many different components. Points = 6 FPs.
\item Add money to the balance: For security reasons, this operation requires some steps. We give it average complexity. Points = 4FPs.
\item Edit personal info: Some validation of the new data is required, therefore it's average complexity. Points = 4FPs
\item Report a damage: this is a simple task, involving only one component. Points = 3FPs.
\end{itemize}
Finally, we list the interaction that an employee can have with the platform:
\begin{itemize}[noitemsep]
\item Drawing a new safe area: Since the process requires the area to be valid (and the validity of an area needs to be checked), we think this is a complex opearation. Points = 6FPs.
\item Taking in charge a vehicle: This is a straightforward task and a fairly simple one. Points = 3FPs.
\item Register/Dismiss a car to the fleet: Since this operation requires cooperation between the car and the central system, we conside it to be complex. Points = 6x2 FPs
\end{itemize}
The following table sums up EIs complexity and FP associated:\\\\
\begin{tabular}{|p{5cm}|p{3cm}|p{2cm}|}
\hline
\textbf{EI} & \textbf{Complexity} & \textbf{FP} \\
\hline
Login & Low & 3\\
Logout & Low & 3\\
Register new account & Average & 4 \\
Rent a car & High & 6 \\
Add money & Average & 4 \\
Edit personal info & Average & 4 \\
Report a damage & Low & 3 \\
Drawing new safe area & High & 6\\
Take in charge vehicle & Low & 3\\
Register a car & High & 6\\
Dismiss a car & High & 6\\
\hline
\multicolumn{2}{|l|}{\textbf{Total}} & 48\\
\hline
\end{tabular}

\subsection{External Inquiries}
The definition of external inquiry is a data request performed by a user. We divide have two different type of users, customers and employees. For the first we have the following possible inquiries:
\begin{itemize}
\item Get personal information
\item Get available vehicles from a given position
\end{itemize}
For the second we have:
\begin{itemize}
\item Get Safe areas
\item Get assistance list
\end{itemize}
Getting the available vehicles requires a bit of computation regarding vehicles position, so it will be labeled with an Average complexity. All the other interactions can be accomplished by a single query each, so they're classified with a Low complexity. A brief recap shown in the following table:\\
\begin{tabular}{|p{6cm}|p{3cm}|p{2cm}|}
\hline
\textbf{External Inquiry} & \textbf{Complexity} & \textbf{FP}\\
\hline
Get personal information & Low & 3\\
Get available vehicles & Average & 4\\
Get safe areas & Low & 3\\
Get assistance list & Low & 3\\
\hline
\multicolumn{2}{|l|}{\textbf{Total}} & 13\\
\hline
\end{tabular}
\subsection{External Outputs}
The system need also to communicate with users other information than results of inquiries. These operations go under the name of External Outputs, and occurr on these occations:
\begin{itemize}
\item Notify a user he's a Debtor
\item Notify a user he's been Banned
\item Notify a user he committed an Infraction
\item Notify a user his payment outcome
\item Notify an employee a car has been added to assistance list
\end{itemize}
Since the Notification Gateway our system is going to use is an external one, all these operations have a low complexity in our development process.
\\
\begin{tabular}{|p{5cm}|p{3cm}|p{2cm}|}
\hline
\textbf{External Outputs} & \textbf{Complexity} & \textbf{FP} \\
\hline

Notify user he's a Debtor & Low & 4\\
Notify user he's been Banned & Low & 4\\
Notify user he committed an Infraction & Low & 4\\
Notify user his payment outcome & Low & 4\\
Notify employee a car has been added to assistance list & Low & 4\\
\hline
\multicolumn{2}{|l|}{\textbf{Total}} & 20\\
\hline
\end{tabular}
\subsection{Overall Estimation}
A recap of everything we have stated about the complexity of our system is provided by this table
\begin{tabular}{|l|l|}
\hline
\textbf{External Outputs} & \textbf{FP} \\
\hline

Internal Logic File & 69\\
External Logic File & TODO\\
External Inputs & TODO\\
External Inquiries & 13\\
External Outputs & 20\\
\hline
\textbf{Total}} & TODO\\
\hline
\end{tabular}
\subsection{Cost and effort estimation: COCOMO II}

\subsubsection{Scale Drivers}

\hspace*{-3cm}\begin{tabular}{|p{3cm}|p{2cm}|p{2cm}|p{2cm}|p{2cm}|p{2cm}|p{2cm}|}
	\hline
	\textbf{Scale Factors} & \textbf{Very Low} & \textbf{Low} & \textbf{Nominal} & \textbf{High} & \textbf{Very High} & \textbf{Extra High}\\
	\hline
	\textbf{PREC} & thoroughly unprecedented & largely unprecedented & somewhat unprecedented & generally familiar & largely familiar & thoroughly familiar\\
	SF$_j$ & 6.20 & 4.96 & 3.72 & 2.48 & 1.24 & 0.00\\\hline
	\textbf{FLEX} & rigorous & occasional relaxation & some relaxation & general conformity & some conformity & general goals\\
	SF$_j$ & 5.07 & 4.05 & 3.04 & 2.03 & 1.01 & 0.00\\\hline
	\textbf{RESL} & little (20\%) & some (40\%) & often (60\%) & generally (75\%) & mostly (90\%) & full (100\%)\\
	SF$_j$ & 7.07 & 5.65 & 4.24 & 2.83 & 1.41 & 0.00\\\hline
	\textbf{TEAM} & very difficult interactions & some difficult interactions & basically cooperative interactions & largely cooperative & highly cooperative & seamless interactions\\
	SF$_j$ & 5.48 & 4.38 & 3.29 & 2.19 & 1.10 & 0.00\\\hline
	\textbf{PMAT} & Level 1 Lower & Level 1 Upper & Level 2 & Level 3 & Level 4 & Level 5\\
	SF$_j$ & 7.80 & 6.24 & 4.68 & 3.12 & 1.56 & 0.00\\\hline
\end{tabular}

\begin{itemize}
\item \textbf{Precedentness}: represents the level of experience our team has in developing large projects. Since this is the first project of such dimension we're facing, this value will be low.
\item \textbf{Development Flexibility}: represents the degree of flexibility in the development process with respect to external specification and requirements. Since the stakeholders do not exist, we assume requirements as immutable, thus the value will be low.
\item \textbf{Risk resolution}: represents how much we are aware of risks and how we are ready to react if anything happens. Since we have performed a good risk analisys but not much extensive, we will set this value to Nominal
\item \textbf{Team Cohesion}: represents the ability of team members to work togheter and know each other. Since we had exceprience in the development of another project last year, this value is set to very high.
\item \textbf{Project Maturity}: represents the level of progress in the system project. Since we have already produced RASD, DD and ITPD, this is set to level 4.
\end{itemize}
Here's a tabular recap of what stated above\\
\begin{tabular}{|p{6cm}|p{3cm}|p{2cm}|}
\hline
\textbf{Scale Driver} & \textbf{Factor} & \textbf{Value}\\
\hline
Precedentness(PREC) & Low & 4.96\\
Development Flexibility(FLEX) & Low & 4.05\\
Risk Resolution(RESL) & Nominal & 4.24\\
Team Cohesion(TEAM) & Very High & 1.10\\
Process Maturity(PMAT) & Level 4 & 1.56\\
\hline
\multicolumn{2}{|l|}{\textbf{Total}} & 15.91\\
\hline
\end{tabular}

\subsubsection{Cost Drivers}

\begin{itemize}
\item \textbf{Required Software Reliability}:\\
Since there are many car sharing services, a malfunctioning would lead to a loss of customers for the company, along with a bad reputation. This can be translated into a high financial lost, so the Rating Level is set to High
\end{itemize}
\hspace*{-3cm}\begin{tabular}{|p{3cm}|p{2cm}|p{2cm}|p{2cm}|p{2cm}|p{2cm}|p{2cm}|}
\hline
\multicolumn{7}{|c|}{\textbf{RELY Cost Drivers}}\\
\hline
\hline
\textbf{Rating Level} & \textbf{Very Low} & \textbf{Low} & \textbf{Nominal} & \textbf{High} & \textbf{Very High} & \textbf{Extra High}\\
\hline
\textbf{Descriptor} & Slightly inconvenience & Easily recoverable losses & Moderate recoverable losses & Highly financial loss & risk to human life & \\
\hline
\textbf{Effort multiplier} & 0.82 & 0.92 & 1.00 & 1.10 & 1.26 & n/a\\
\hline 
\end{tabular}
\begin{itemize}
\item \textbf{Database size}:\\
Given the aforementioned tables and attributes, we estimate roughly the dimension of database to test the system, to be around 200MB: since we estimate KSLOC to be between 8 and 11, the ratio D/P is between 18 and 26. The Rating Level is set to Nominal.
\end{itemize}
\hspace*{-3cm}\begin{tabular}{|p{3cm}|p{2cm}|p{2cm}|p{2cm}|p{2cm}|p{2cm}|p{2cm}|}
\hline
\multicolumn{7}{|c|}{\textbf{DATA Cost Drivers}}\\
\hline
\hline
\textbf{Rating Level} & \textbf{Very Low} & \textbf{Low} & \textbf{Nominal} & \textbf{High} & \textbf{Very High} & \textbf{Extra High}\\
\hline
\textbf{Descriptor} & & $D/P < 10$ & $10 \leq D/P \leq 100$ & $100 \leq D/P \leq 1000$ & $ D/P > 100$ & \\
\hline
\textbf{Effort multiplier} & n/a & 0.90 & 1.00 & 1.14 & 1.28 & n/a\\
\hline 
\end{tabular}
\begin{itemize}
\item \textbf{Product Complexity}:\\
According to COCOMO II CPLX table, this Rating Level is set to Nominal.
\end{itemize}
\hspace*{-3cm}\begin{tabular}{|p{3cm}|p{2cm}|p{2cm}|p{2cm}|p{2cm}|p{2cm}|p{2cm}|}
\hline
\multicolumn{7}{|c|}{\textbf{CPLX Cost Drivers}}\\
\hline
\hline
\textbf{Rating Level} & \textbf{Very Low} & \textbf{Low} & \textbf{Nominal} & \textbf{High} & \textbf{Very High} & \textbf{Extra High}\\
\hline
\textbf{Effort multiplier} & 0.73 & 0.87 & 1.00 & 1.17 & 1.34 & 1.74\\
\hline 
\end{tabular}
\begin{itemize}
\item \textbf{Required Reusability}:\\ the company might want to extend the car sharing service to other types of vehicle, so some components should be developed as reusable but only in the project scope. The Rating Level is thus set to Nominal
\end{itemize}
\hspace*{-3cm}\begin{tabular}{|p{3cm}|p{2cm}|p{2cm}|p{2cm}|p{2cm}|p{2cm}|p{2cm}|}
\hline
\multicolumn{7}{|c|}{\textbf{RUSE Cost Driver}}\\
\hline
\hline
\textbf{Rating Level} & \textbf{Very Low} & \textbf{Low} & \textbf{Nominal} & \textbf{High} & \textbf{Very High} & \textbf{Extra High}\\
\hline
\textbf{Descriptor} &  & None & Across project & Across program & Across product line & Across multiple product lines\\
\hline
\textbf{Effort multiplier} & n/a & 0.95 & 1.00 & 1.07 & 1.15 & 1.26\\
\hline 
\end{tabular}
\begin{itemize}
\item \textbf{Documentation match to life-cycles}:\\
Every document needed for product life-cycles has been developed, so we set this Rating Level to Nominal
\end{itemize}
\hspace*{-3cm}\begin{tabular}{|p{3cm}|p{2cm}|p{2cm}|p{2cm}|p{2cm}|p{2cm}|p{2cm}|}
\hline
\multicolumn{7}{|c|}{\textbf{DOCU Cost Drivers}}\\
\hline
\hline
\textbf{Rating Level} & \textbf{Very Low} & \textbf{Low} & \textbf{Nominal} & \textbf{High} & \textbf{Very High} & \textbf{Extra High}\\
\hline
\textbf{Descriptor} & Many life-cycle needs uncovered & Some life-cycle needs uncovered & Right sized to life-cycle needs uncovered & Excessive for life-cycle needs uncovered & Very excessive for life-cycle needs uncovered & \\
\hline
\textbf{Effort multiplier} & 0.81 & 0.91 & 1.00 & 1.11 & 1.23 & n/a\\
\hline 
\end{tabular}
\begin{itemize}
\item \textbf{Execution timconstraint}:\\
As PowerEnJoy will envolve a fairly usage of the CPU available time, due to both the complexity of the system and the high level of traffic generated by both users and cars, the Rating Level is set to Very High
\end{itemize}
\hspace*{-3cm}\begin{tabular}{|p{3cm}|p{2cm}|p{2cm}|p{2cm}|p{2cm}|p{2cm}|p{2cm}|}
\hline
\multicolumn{7}{|c|}{\textbf{TIME Cost Driver}}\\
\hline
\hline
\textbf{Rating Level} & \textbf{Very Low} & \textbf{Low} & \textbf{Nominal} & \textbf{High} & \textbf{Very High} & \textbf{Extra High}\\
\hline
\textbf{Descriptor} &  &  & Less than 50\% use of available execution time & 70\% use of available execution time & 85\% use of available execution time & 90\% use of available execution time\\
\hline
\textbf{Effort multiplier} & n/a & n/a & 1.00 & 1.11 & 1.29 & 1.63\\
\hline 
\end{tabular}
\begin{itemize}
\item \textbf{Storage Constraint}: \\Current disk drives offer a more than suitable capacity for the need of our application. However, an increase in the number of vehicles could require more and more data to be stored, so we overestimate this Rating Level to High
\end{itemize}
\hspace*{-3cm}\begin{tabular}{|p{3cm}|p{2cm}|p{2cm}|p{2cm}|p{2cm}|p{2cm}|p{2cm}|}
\hline
\multicolumn{7}{|c|}{\textbf{STOR Cost Driver}}\\
\hline
\hline
\textbf{Rating Level} & \textbf{Very Low} & \textbf{Low} & \textbf{Nominal} & \textbf{High} & \textbf{Very High} & \textbf{Extra High}\\
\hline
\textbf{Descriptor} &  &  & $\leq 50 \%$ use of available storage & 70\% use of available storage & 85\% use of available storage& 90\% use of available storage\\
\hline
\textbf{Effort multiplier} & n/a & n/a & 1.00 & 1.05 & 1.17 & 1.46\\
\hline 
\end{tabular}
\begin{itemize}
\item \textbf{Platform Volatility}:\\
We don't expect major changes to happen often, as the additional functionalities for such an application are limited; plus, as we don't plan on providing a dedicated app, even client side won't need much changes to remain up to date. For these reasons, the Rating Level is set to Low
\end{itemize}
\hspace*{-3cm}\begin{tabular}{|p{3cm}|p{2cm}|p{2cm}|p{2cm}|p{2cm}|p{2cm}|p{2cm}|}
\hline
\multicolumn{7}{|c|}{\textbf{PVOL Cost Driver}}\\
\hline
\hline
\textbf{Rating Level} & \textbf{Very Low} & \textbf{Low} & \textbf{Nominal} & \textbf{High} & \textbf{Very High} & \textbf{Extra High}\\
\hline
\textbf{Descriptor} & Major changes every 12 months, minor changes every 1 month & Major changes every 6 months, minor changes every 2 weeks & Major changes every 2 months, minor changes every 1 week & Major changes every 2 weeks, minor changes every 2 days &  & \\
\hline
\textbf{Effort multiplier} & n/a & 0.87 & 1.00 & 1.15 & 1.30 & n/a\\
\hline 
\end{tabular}
\begin{itemize}
\item \textbf{Analyst Capability}:\\
we have deeply analized the problem and its real world impact, so we set this Rating Level to High.
\end{itemize}
\hspace*{-3cm}\begin{tabular}{|p{3cm}|p{2cm}|p{2cm}|p{2cm}|p{2cm}|p{2cm}|p{2cm}|}
\hline
\multicolumn{7}{|c|}{\textbf{ACAP Cost Drivers}}\\
\hline
\hline
\textbf{Rating Level} & \textbf{Very Low} & \textbf{Low} & \textbf{Nominal} & \textbf{High} & \textbf{Very High} & \textbf{Extra High}\\
\hline
\textbf{Descriptor} & 15th percentile & 35th percentile & 55th percentile & 75th percentile & 90th percentile & \\
\hline
\textbf{Effort multiplier} & 1.42 & 1.19 & 1.00 & 0.85 & 0.71 & n/a\\
\hline 
\end{tabular}
\begin{itemize}
\item \textbf{Programmer Capability}:\\
The project must still be developed, but we have already worked together as a team, so we are confident in our capabilities. We will set this parameter to High.
\end{itemize}
\hspace*{-3cm}\begin{tabular}{|p{3cm}|p{2cm}|p{2cm}|p{2cm}|p{2cm}|p{2cm}|p{2cm}|}
\hline
\multicolumn{7}{|c|}{\textbf{PCAP Cost Driver}}\\
\hline
\hline
\textbf{Rating Level} & \textbf{Very Low} & \textbf{Low} & \textbf{Nominal} & \textbf{High} & \textbf{Very High} & \textbf{Extra High}\\
\hline
\textbf{Descriptor} & 15th percentile & 35th percentile & 55th percentile & 75th percentile & 90th percentile & \\
\hline
\textbf{Effort multiplier} & 1.34 & 1.15 & 1.00 & 0.88 & 0.76 & n/a\\
\hline 
\end{tabular}

\begin{itemize}
\item \textbf{Application Experience}:\\
This is the first time we are using JEE, even if we have already worked on some java applications. We are setting this parameter to Very Low, since we have started using JEE less than 2 months ago.
\end{itemize}
\hspace*{-3cm}\begin{tabular}{|p{3cm}|p{2cm}|p{2cm}|p{2cm}|p{2cm}|p{2cm}|p{2cm}|}
\hline
\multicolumn{7}{|c|}{\textbf{APEX Cost Driver}}\\
\hline
\hline
\textbf{Rating Level} & \textbf{Very Low} & \textbf{Low} & \textbf{Nominal} & \textbf{High} & \textbf{Very High} & \textbf{Extra High}\\
\hline
\textbf{Descriptor} & less than 2 months & 6 months & 1 year & 3 years & 6 years & \\
\hline
\textbf{Effort multiplier} & 1.22 & 1.10 & 1.00 & 0.88 & 0.81 & n/a\\
\hline 
\end{tabular}

\begin{itemize}
\item \textbf{Platform Experience}:\\
As already said, we have little experience with JEE. We have however some experience in databases and graphic user interfaces, so we are setting this parameter to Nominal.
\end{itemize}
\hspace*{-3cm}\begin{tabular}{|p{3cm}|p{2cm}|p{2cm}|p{2cm}|p{2cm}|p{2cm}|p{2cm}|}
\hline
\multicolumn{7}{|c|}{\textbf{PLEX Cost Driver}}\\
\hline
\hline
\textbf{Rating Level} & \textbf{Very Low} & \textbf{Low} & \textbf{Nominal} & \textbf{High} & \textbf{Very High} & \textbf{Extra High}\\
\hline
\textbf{Descriptor} & less than 2 months & 6 months & 1 year & 3 years & 6 years & \\
\hline
\textbf{Effort multiplier} & 1.19 & 1.09 & 1.00 & 0.91 & 0.85 & n/a\\
\hline 
\end{tabular}

\begin{itemize}
\item \textbf{Language and Tool Experience}:\\
For the reasons stated above, this parameter is set to Nominal
\end{itemize}
\hspace*{-3cm}\begin{tabular}{|p{3cm}|p{2cm}|p{2cm}|p{2cm}|p{2cm}|p{2cm}|p{2cm}|}
\hline
\multicolumn{7}{|c|}{\textbf{LTEX Cost Driver}}\\
\hline
\hline
\textbf{Rating Level} & \textbf{Very Low} & \textbf{Low} & \textbf{Nominal} & \textbf{High} & \textbf{Very High} & \textbf{Extra High}\\
\hline
\textbf{Descriptor} & less than 2 months & 6 months & 1 year & 3 years & 6 years & \\
\hline
\textbf{Effort multiplier} & 1.20 & 1.09 & 1.00 & 0.91 & 0.84 & n/a\\
\hline 
\end{tabular}

\begin{itemize}
\item \textbf{Personnel Continuity}:\\
Since before being programmer we are student, we can spend only part of our day in the development of this project. For this reason, this parameter is set to Very Low.
\end{itemize}
\hspace*{-3cm}\begin{tabular}{|p{3cm}|p{2cm}|p{2cm}|p{2cm}|p{2cm}|p{2cm}|p{2cm}|}
\hline
\multicolumn{7}{|c|}{\textbf{PCON Cost Driver}}\\
\hline
\hline
\textbf{Rating Level} & \textbf{Very Low} & \textbf{Low} & \textbf{Nominal} & \textbf{High} & \textbf{Very High} & \textbf{Extra High}\\
\hline
\textbf{Descriptor} & 48 \% /year  & 24 \% /year & 12 \% /year & 6 \% /year & 3 \% /year & \\
\hline
\textbf{Effort multiplier} & 1.29 & 1.12 & 1.00 & 0.90 & 0.81 & n/a\\
\hline 
\end{tabular}

\begin{itemize}
\item \textbf{Usage of Software Tools}:\\
Our application platform supports basic life-cycle tools, so we will set this parameter to Nominal
\end{itemize}
\hspace*{-3cm}\begin{tabular}{|p{3cm}|p{2cm}|p{2cm}|p{2cm}|p{2cm}|p{2cm}|p{2cm}|}
\hline
\multicolumn{7}{|c|}{\textbf{TOOL Cost Driver}}\\
\hline
\hline
\textbf{Rating Level} & \textbf{Very Low} & \textbf{Low} & \textbf{Nominal} & \textbf{High} & \textbf{Very High} & \textbf{Extra High}\\
\hline
\textbf{Descriptor} & edit, code, debug & simple, frontend, backend CASE, little integration & basic life-cycle tools, moderately integrate  & strong, mature life-cycle tools, moderately integrated & strong, mature, proactive life-cycle tools, well integrated with processes, methods, reuse & \\
\hline
\textbf{Effort multiplier} & 1.17 & 1.09 & 1.00 & 0.90 & 0.78 & n/a\\
\hline 
\end{tabular}

\begin{itemize}
\item \textbf{Multisite Development}:\\
Even if we work from two different cities, we have frequent wideband electronic communication and occasional voice conference. Due to these factors, we are setting this parameter to Very High.
\end{itemize}
\hspace*{-3cm}\begin{tabular}{|p{3cm}|p{2cm}|p{2cm}|p{2cm}|p{2cm}|p{2cm}|p{2cm}|}
\hline
\multicolumn{7}{|c|}{\textbf{SITE Cost Driver}}\\
\hline
\hline
\textbf{Rating Level} & \textbf{Very Low} & \textbf{Low} & \textbf{Nominal} & \textbf{High} & \textbf{Very High} & \textbf{Extra High}\\
\hline
\textbf{Descriptor} & Some phone, mail & Individual phone, FAX & Narrowband email & Wideband electronic communication & Wideband elect. comm, occasional video conf. & Interactive multimedia\\
\hline
\textbf{Effort multiplier} & 1.22 & 1.09 & 1.00 & 0.93 & 0.86 & 0.80\\
\hline 
\end{tabular}

\begin{itemize}
\item \textbf{Required Development Schedule}:\\
Since we plan to put more effort in the early part of our project, we will set this parameter to High
\end{itemize}
\hspace*{-3cm}\begin{tabular}{|p{3cm}|p{2cm}|p{2cm}|p{2cm}|p{2cm}|p{2cm}|p{2cm}|}
\hline
\multicolumn{7}{|c|}{\textbf{SCED Cost Driver}}\\
\hline
\hline
\textbf{Rating Level} & \textbf{Very Low} & \textbf{Low} & \textbf{Nominal} & \textbf{High} & \textbf{Very High} & \textbf{Extra High}\\
\hline
\textbf{Descriptor} & 75 \% of nominal & 85 \% of nominal & 100 \% of nominal & 130 \% of nominal & 160 \% of nominal & \\
\hline
\textbf{Effort multiplier} & 1.43 & 1.14 & 1.00 & 1.00 & 1.00 & n/a\\
\hline 
\end{tabular}


We can resume cost drivers in this table\\
\begin{tabular}{|p{8cm}|p{3cm}|p{2cm}|}
\hline
\textbf{Cost Driver} & \textbf{Factor} & \textbf{Value}\\
\hline
Required Software Reliability(RELY) & High & 1.10\\
Database size(DATA) & High & 1.14\\
Product Complexity(CPLX) & Nominal & 1.00\\
Required Reusability(RUSE) & Nominal & 1.00\\
Documentation match to life-cycles(DOCU) & Nominal & 1.00\\
Execution time constraint(TIME) & Very High & 1.29\\
Storage Constraint(STOR) & High & 1.05\\
Platform Volatility(PVOL) & Low & 0.87\\
Analyst Capability(ACAP) & High & 0.85\\
Programmer Capability(PCAP) & High & 0.88\\
Application Experience(APEX) & Very Low & 1.22\\
Platform Experience(PLEX) & Nominal & 1.00\\
Language and Tool Experience(LTEX) & Nominal & 1.00\\
Personnel Continuity(PCON) & Very Low & 1.29\\
Usage of Software Tools(TOOL) & Nominal & 1.00\\
Multisite Development(SITE) & Very High & 0.86\\
Required Development Schedule(SCED) & High & 1.00\\
\hline
\multicolumn{2}{|l|}{\textbf{Total}} & 1.496\\
\hline
\end{tabular}

\subsubsection{Effort Equation}
We are now going to calculate the effort expressed in Person-Months (PM)\\

$$\displaystyle Effort = A \times EAF \times KSLOC^E $$ \\
In COCOMO II, A has the following value:\\
$$\displaystyle A = 2.94 $$ \\
EAF has the precedently calculated value:\\
$$\displaystyle EAF = 1.496 $$ \\
Finally, let's calculate E (B=0.91 in COCOMO II):
$$\displaystyle E = B + 0.01 \times \sum_{i}SF_i = B + 0.01 \times 15.91 = 0.91 + 0.1591 = 1.0691  $$ \\
We can now calculate the Effort lower bound: \\
$$\displaystyle Effort = 2.94 \times 1.496 \times 7.682^E = 38.899 PM \approx 39PM $$ \\
And upper bound:\\
$$\displaystyle Effort = 2.94 \times 1.496 \times 11.189^E = 58.148 PM \approx 59PM $$ \\
\subsection{Schedule Estimation}
We can now estimate the project duration:\\
$$\displaystyle Duration = 3.67 \times Effort^F $$ \\
Let's calculate F:\\
$$\displaystyle F = 0.28 + 0.2 \times (E - B) = 0.28 + 0.2\times (1.0691 - 0.91) = 0.28 + 0.2\times 0.1591 = 0.31182 $$
Finally, we have a lower bound of:\\
$$\displaystyle Duration = 3.67 \times 38.899^F = 11.49 months $$ 
And an upper bound of:\\
$$\displaystyle Duration = 3.67 \times 58.148^F = 13.02 months $$ 
\newpage
\section{Schedule}
We're going to present the first draft of the scheduling we would follow to develop this software. We tried to be as responding as possibile in the scheduling of what we have really done, that is RASD and DD development. For what concerns the successive phases, as we have never approached such a complex software, we made a rough estimation of the time each activity would take. As we gain experience along the real development, we could provide more precise estimation. We followed an overall waterfall approach.\\
We splitted images to enhance readibility, presenting the scheduling of each macro-activity.\\

\begin{figure}
\centering
\vspace*{-4cm}\includegraphics[scale=0.7]{Gantt_RASD}
\end{figure}
\begin{figure}
\centering
\vspace*{-4cm}\includegraphics[scale=0.7]{Gantt_DD}
\end{figure}
\begin{figure}
\centering
\vspace*{-4cm}\includegraphics[scale=0.7]{Gantt_Development}
\end{figure}
\begin{figure}
\centering
\vspace{-4cm}\includegraphics[scale=0.6]{Gantt_Deployment_Startup}
\end{figure}
\newpage
\section{Resource Allocation}

As our team is composed only of two people, we will mostly collaborate on every task with few exceptions, thus each of us is allocated to each task. This will also help as we have limited time to spend on the project, so we could easily continue and complete each other's tasks in case of need.\\
\begin{figure}
\hspace*{-4cm}\includegraphics[width = \paperwidth]{Gantt_Boriero}
\end{figure}

\begin{figure}
\hspace*{-4cm}\includegraphics[width = \paperwidth]{Gantt_Brunitti}
\end{figure}

\newpage
\section{Risk Managment}
\begin{enumerate}

\item \textbf{Burocracy issues}:\\
One of the most dangerous risks is related to the interaction with local government. The acquiring of permissions could take a long time, depending on the city, and this could lead in a major delay in our project. To prevent this we overestimate the time needed to obtain those permissions.
\item \textbf{Data loss}:\\
Data loss can be a consequence to hackers' attack or power supply issues. To avoid this problem we need to backup our data frequently on multiple external disks.
\item \textbf{Stakeholders' bankrupt}:\\
Since we have been commissioned this project by a private company and all the costs are covered by them, a bankrupt of that company would mean an almost certain failure for our software company. To prevent this, we ask them a financial guarantee.
\item \textbf{Dependency on external services}:\\
As we exploit external services in our project, like GoogleMaps, Payment services and Notification services, we might encounter compatibility problems if any of the external provider introduces major changes in the way services are provided.
A kind of preventive solution could involve exploiting information hiding and designing components in a modular way, in order to isolate the interface with those external services and make it easier to replace.
\item \textbf{Changes in requirememnts}:\\
Following a meeting with the stakeholders, some controversy concerning our interpretation of the requirements might arise. To avoid to make major changes to our project, we will schedule many meetings with the stakeholder in order to minimize incomprehensions by keeping a constant flow of information between our software house and the company.
\item \textbf{Low appeal to user}:\\
Once the software is released, we will incurr in user's review of our pruduct. To address better their taste, we will release beta versions of the client side during the development process, in order to collect opinions and adjust our work prior to actual releas. This will prevent us from applyng big changes right after the actual release, and will help us to develop a better product.
\end{enumerate}
\newpage
\section{Effort Spent}
Brunitti Simone:
\begin{itemize}
\item 6/6/6: 666 hours
\end{itemize}
Boriero Stefano:
\begin{itemize}
\item 16/01	1.30 hours
\item 17/01	4 hours
\item 18/01	4 hours
\item 19/01	1.30 hours
\item Total 11 hours
\end{itemize}
\end{document}
