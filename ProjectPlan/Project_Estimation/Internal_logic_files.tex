\subsubsection{Internal Logic Files}
In order to provide its functionalities, PowerEnjoy will need to store data and information.\\
The main information it need to store is about vehicles. The relevant attributes needed are 
\begin{itemize}
\item Vehicle Id
\item Plate number
\item Status
\item BatteryLevel
\item Position
\end{itemize}

Another fundamental table is the one about rides, containing this information:
\begin{itemize}
\item Starting point
\item Starting battery
\item Number of passengers
\item Final position
\item Final battery
\end{itemize}

To complete the set of tables oriented to the actual sharing functionality, the system will have a Reservation table with the following information:
\begin{itemize}
\item User id
\item Vehicle id
\item Starting time
\item Unlocked car
\end{itemize}

For assistance list related functionalities the system needs an assistance table to store:
\begin{itemize}
\item Vehicle id
\item Fault
\item Status
\item Employee in charge
\end{itemize}

Employee's information are stored as well:
\begin{itemize}
\item Employee id
\item Name
\item Surname
\item District
\item Job
\end{itemize}

Then the system needs all usual information about users:
\begin{itemize}
\item User id
\item Username
\item Name
\item Surname
\item Driving license
\end{itemize}

For security issues, payment information are stored in a different table:
\begin{itemize}
\item User id
\item Payment info
\end{itemize}


Safe areas are store in a two-level structure. The first one containig:
\begin{itemize}
\item SafeArea id
\item Type
\end{itemize}


The second one containing:
\begin{itemize}
\item SafeArea id
\item Position
\end{itemize}


The following table sums up ILF's complexity and FP associated
\\
\begin{tabular}{|p{5cm}|p{3cm}|p{2cm}|}
\hline
\textbf{ILF} & \textbf{Complexity} & \textbf{FP} \\
\hline

Login data & Low & 7\\
User data & Low & 7\\
Employee data & Low & 7\\
Password & Average & 10\\
Safe areas & Low & 7\\
Vehicle & Low & 7\\
Reservation & Low & 7\\
Ride & Average & 10\\
Assistance & Low & 7\\
\hline
\multicolumn{2}{|l|}{\textbf{Total}} & 69\\
\hline
\end{tabular}