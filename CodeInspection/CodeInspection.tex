\documentclass{article}
\usepackage{enumitem}
\usepackage[T1]{fontenc}
\usepackage{hyperref}

\begin{document}
\section{Introduction}
This document aims to present the outcome of a code inspection perfomed over the ServiceEventHandler class of Apache OfBiz project. The class can be found at\\
\textbf{apache-ofbiz-16.11.01/apache-ofbiz-16.11.01/framework/webapp/src/}\\
\textbf{main/java/org/apache/ofbiz/webapp/event/ServiceEventHandler.java}\\

\section{Functional Role}
The main functionality of this class is to invoke a service. In particular, this class invokes a service whose request has been triggered by an event. To do that, this class checks that all the parameters are well formed, and sets up a map filled with those parameters. The definitions of such maps have been encoded in XML files that are located at\\
\textbf{apache-ofbiz-16.11.01/framework/service/servicedef/services.xml}\\

A brief explanation on how this xml maps are used and how the service framework works can be found at \href{https://cwiki.apache.org/confluence/display/OFBIZ/Service+Engine+Guide#ServiceEngineGuide-serviceDefinition}{\textbf{here}}

\section{Checklist}
\subsection{Naming Conventions}
Nothing to say
\subsection{Indention}
Nothing to say
\subsection{Braces}
There are three if statements followed by one instruction not included between braces at lines 106, 253, 394
\subsection{File Organization}

\subsection{Wrapping Lines}

\subsection{Comments}
There's a commented out block of code in lines 180-184, whithout any explanation

\subsection{Java Source Files}
Javadoc is minimal. It describes only the main goal and doesn't help to fully understand the way it's achieved. There's no javadoc for the static method, but it is widely commented.

\subsection{Package and Import Statements}
Nothing to say about that

\subsection{Class and Interface Declarations}
The invoke method is too long. As it deals mostly with multiple checks on inputs and environmental setup, it could be splitted in different subroutines

\subsection{Initialization and Declarations}
The variable serviceName is delcared and initialized to Null on line 93, but is later set to event.invoke, on line 102. We suggest to directly intialize serviceName to event.invoke.\\
Many declarations appear in the middle of a code block. This may be due to the fact the the invoke method is too long. A possible solution to this problem could be using private methods, making the whole class more readable.\\
Here are the lines and the declarations that are not compatible with this rule:
\begin{itemize}[noitemsep]
\item Line 86:
\begin{itemize}[noitemsep]
\item DispatchContext dctx = dispatcher.getDispatchContext();\\
\end{itemize}
\item Line 92 and 93:
\begin{itemize}[noitemsep]
\item String mode = SYNC;
\item String serviceName = null; 
\end{itemize}   
\item Line from 109 to 112:
\begin{itemize}[noitemsep]
\item Locale locale = UtilHttp.getLocale(request);
\item TimeZone timeZone = UtilHttp.getTimeZone(request);
\item HttpSession session = request.getSession();
\item GenericValue userLogin = (GenericValue) session.getAttribute("userLogin"); 
\end{itemize} 
\item Line 115:
\begin{itemize}[noitemsep]
\item ModelService model = null;
\end{itemize} 
\item Line 132: 
\begin{itemize}[noitemsep]
\item boolean isMultiPart = ServletFileUpload.isMultipartContent(request);
\end{itemize}  
\item Line 133:
\begin{itemize}[noitemsep]
\item Map<String, Object> multiPartMap = new HashMap<String, Object>();
\end{itemize}
\item Line 146 and 147:
\begin{itemize}[noitemsep]
\item  String sizeThresholdStr = EntityUtilProperties.getPropertyValue("general", "http.upload.max.sizethreshold", "10240", dctx.getDelegator());
\item int sizeThreshold = 10240; 
\end{itemize}  
\item Line 155 and 156: 
\begin{itemize}[noitemsep]
\item String tmpUploadRepository = EntityUtilProperties.getPropertyValue("general", "http.upload.tmprepository", "runtime/tmp", dctx.getDelegator());
\item String encoding = request.getCharacterEncoding(); 
\end{itemize}         
\item Line 159: 
\begin{itemize}[noitemsep]
\item ServletFileUpload upload = new ServletFileUpload(new DiskFileItemFactory(sizeThreshold, new File(tmpUploadRepository))); \\
\end{itemize}  
\item Line 162: 
\begin{itemize}[noitemsep]
\item FileUploadProgressListener listener = new FileUploadProgressListener();
\end{itemize}  
\item Line 171:
\begin{itemize}[noitemsep]
\item  List<FileItem> uploadedItems = null;
\end{itemize}
\item Line 234 and 235:
\begin{itemize}[noitemsep]
\item  Map<String, Object> rawParametersMap = UtilHttp.getCombinedMap(request);
\item Set<String> urlOnlyParameterNames = UtilHttp.getUrlOnlyParameterMap(request).keySet();
\end{itemize}
\item Line 238:
\begin{itemize}[noitemsep]
\item Map<String, Object> serviceContext = new HashMap<String, Object>();
\end{itemize} 
\item Line 249: 
\begin{itemize}[noitemsep]
\item Object value = null;
\end{itemize}
\item Line 312:
\begin{itemize}[noitemsep]
\item List<Object> errorMessages = new LinkedList<Object>();
\end{itemize}
\item Line 336: 
\begin{itemize}[noitemsep]
\item Map<String, Object> result = null;
\end{itemize}
\item Line 361:
\begin{itemize}[noitemsep]
\item String responseString = null;
\end{itemize}
\end{itemize}
\subsection{Output Format}
Some outputs signlaing errors throughout the code are a bit general. Maybe the author should provide additional information on these problems. These outputs can be found at:
\begin{itemize}[noitemsep]
\item Line 120 and 124: the output displays a "problems getting the service model" message
\item Line 175: the output displays a "problem reading uploaded data" message
\item Line 357 and 358: the output is a generic "service invocation error" message
\end{itemize}
Moreover, from line 400 to 411 the output displays a long message. We suggest to use "\textbackslash n" to make the output error more readable
\subsection{Exceptions}
Line 151 and 152 contain an inconsistency. The error message clearly states "Unable to obtain the threshold size from general.properties; using default 10K", implying that sizeThreshold will be set to 10240. On line 152, however, we find that sizeThreshold is set to -1. Either the error message or the assignment at line 152 is wrong.\\

\section{Checklist}
\subsection*{Naming Conventions}
\begin{enumerate}
\item All class names, interface names, method names, class variables, method variables, and constants used should have meaningful names and do what the name suggests.
\item If one-character variables are used, they are used only for temporary “throwaway” variables, such as those used in for loops.
\item Class names are nouns, in mixed case, with the first letter of each word in capitalized.
\item Interface names should be capitalized like classes.
\item Method names should be verbs, with the first letter of each addition word capitalized.
\item Class variables, also called attributes, are mixed case, but might begin with an underscore (`\texttt{\_}') followed by a lowercase first letter. All the remaining words in the variable name have their first letter capitalized.
\item Constants are declared using all uppercase with words separated by an underscore.
\end{enumerate}

\subsection*{Indention}\begin{enumerate}[resume]
\item Three or four spaces are used for indentation and done so consistently.
\item No tabs are used to indent.
\end{enumerate}

\subsection*{Braces}\begin{enumerate}[resume]
\item Consistent bracing style is used, either the preferred “Allman” style (first brace goes underneath the opening block) or the “Kernighan and Ritchie” style (first brace is on the same line of the instruction that opens the new block).
\item All \texttt{if}, \texttt{while}, \texttt{do-while}, \texttt{try-catch}, and \texttt{for} statements that have only one statement to execute are surrounded by curly braces.
\end{enumerate}

\subsection*{File Organization}\begin{enumerate}[resume]
\item Blank lines and optional comments are used to separate sections (beginning comments, package/import statements, class/interface declarations which include class variable/attributes declarations, constructors, and methods).
\item Where practical, line length does not exceed 80 characters.
\item When line length must exceed 80 characters, it does NOT exceed 120 characters.
\end{enumerate}

\subsection*{Wrapping Lines}\begin{enumerate}[resume]
\item Line break occurs after a comma or an operator.
\item Higher-level breaks are used.
\item A new statement is aligned with the beginning of the expression at the same level as the previous line.
\end{enumerate}

\subsection*{Comments}\begin{enumerate}[resume]
\item Comments are used to adequately explain what the class, interface, methods, and blocks of code are doing.
\item Commented out code contains a reason for being commented out and a date it can be removed from the source file if determined it is no longer needed.
\end{enumerate}

\subsection*{Java Source Files}\begin{enumerate}[resume]
\item Each Java source file contains a single public class or interface.
\item The public class is the first class or interface in the file.
\item External program interfaces are implemented consistently with what is described in the javadoc.
\item Javadoc is complete (i.e., it covers all classes and files part of the set of classes assigned to you).
\end{enumerate}

\subsection*{Package and Import Statements}\begin{enumerate}[resume]
\item If any package statements are needed, they should be the first non-comment statements. Import statements follow.
\end{enumerate}

\subsection*{Class and Interface Declarations}\begin{enumerate}[resume]
\item The class or interface declarations shall be in the following order:
	\begin{enumerate}
		\item class/interface documentation comment;
		\item class or interface statement;
		\item class/interface implementation comment, if necessary;
		\item class (static) variables;
		\begin{enumerate}
			\item first public class variables;
			\item next protected class variables;
			\item next package level (no access modifier);
			\item last private class variables.
		\end{enumerate}
		\item instance variables;
		\begin{enumerate}
			\item first public instance variables;
			\item next protected instance variables;
			\item next package level (no access modifier);
			\item last private instance variables.
		\end{enumerate}
		\item constructors;
		\item methods.
	\end{enumerate}
	\item Methods are grouped by functionality rather than by scope or accessibility.
	\item Check that the code is free of duplicates, long methods, big classes, breaking encapsulation, as well as if coupling and cohesion are adequate.
\end{enumerate}

\subsection*{Initialization and Declarations}\begin{enumerate}[resume]
\item Variables and class members are of the correct type. Check that they have the right visibility (public/private/protected).
\item Variables are declared in the proper scope.
\item Constructors are called when a new object is desired.
\item All object references are initialized before use.
\item Variables are initialized where they are declared, unless dependent upon a computation.
\item Declarations appear at the beginning of blocks (block: any code surrounded by curly braces `\texttt{\{}' and `\texttt{\}}'). The exception is a variable can be declared in a \texttt{for} loop.
\end{enumerate}

\subsection*{Method Calls}\begin{enumerate}[resume]
\item Parameters are presented in the correct order.
\item Check that the correct method is being called, or should it be a different method with a similar name.
\item Method returned values are used properly.
\end{enumerate}

\subsection*{Arrays}\begin{enumerate}[resume]
\item There are no off-by-one errors in array indexing (that is, all required array elements are correctly accessed through the index).
\item All array (or other collection) indexes have been prevented from going out-of-bounds.
\item Constructors are called when a new array item is desired.
\end{enumerate}

\subsection*{Object Comparison}\begin{enumerate}[resume]
\item All objects (including Strings) are compared with \texttt{equals} and not with \texttt{==}.
\end{enumerate}

\subsection*{Output Format}\begin{enumerate}[resume]
\item Displayed output is free of spelling and grammatical errors.
\item Error messages are comprehensive and provide guidance as to how to correct the problem.
\item The output is formatted correctly in terms of line stepping and spacing.
\end{enumerate}

\subsection*{Computation, Comparisons and Assignments}\begin{enumerate}[resume]
\item Implementation avoids “brutish programming”\footnote{See \url{http://users.csc.calpoly.edu/~jdalbey/SWE/CodeSmells/bonehead.html} for more information.}. 
\item Check order of computation/evaluation, operator precedence and parenthesizing.
\item Liberal use of parenthesis is used to avoid operator precedence problems.
\item All denominators of a division are prevented from being zero.
\item Integer arithmetic, especially division, is used appropriately to avoid causing unexpected truncation/rounding.
\item Comparison and Boolean operators are correct.
\item Check throw-catch expressions, and check that the error condition is actually legitimate.
\item Check that the code is free of any implicit type conversions.
\end{enumerate}

\subsection*{Exceptions}\begin{enumerate}[resume]
\item Check that the most relevant exceptions are caught.
\item Appropriate action are taken for each catch block.
\end{enumerate}

\subsection*{Flow of Control}\begin{enumerate}[resume]
\item In a \texttt{switch} statement, check that all cases are addressed by break or return.
\item All switch statements have a default branch.
\item All loops are correctly formed, with the appropriate initialization, increment and termination expressions.
\end{enumerate}

\subsection*{Files}\begin{enumerate}[resume]
\item All files are properly declared and opened.
\item All files are closed properly, even in the case of an error.
\item EOF conditions are detected and handled correctly.
\item All file exceptions are caught and dealt with accordingly.
\end{enumerate}

\end{document}
